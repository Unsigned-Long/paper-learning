\documentclass[12pt, twocolumn]{article}

% 引入相关的包
\usepackage{amsmath, listings, fontspec, geometry, graphicx, ctex, color, subfigure}

% 设定页面的尺寸和比例
\geometry{left = 1.5cm, right = 1.5cm, top = 1.5cm, bottom = 1.5cm}

% 设定两栏之间的间距
\setlength\columnsep{1cm}

% 设定字体,为代码的插入作准备
\newfontfamily\ubuntu{Ubuntu Mono}

% 头部信息
\title{\normf{这是标题}}
\author{\normf{陈烁龙}}
\date{\normf{\today}}

% 代码块的风格设定
\lstset{
	language=C++,
	basicstyle=\scriptsize\ubuntu,
	keywordstyle=\textbf,
	stringstyle=\itshape,
	commentstyle=\itshape,
	numberstyle=\scriptsize\ubuntu,
	showstringspaces=false,
	numbers=left,
	numbersep=8pt,
	tabsize=2,
	frame=single,
	framerule=1pt,
	columns=fullflexible,
	breaklines,
	frame=shadowbox, 
	backgroundcolor=\color[rgb]{0.97,0.97,0.97}
}

% 字体族的定义
\newcommand{\normf}{\kaishu}
\newcommand{\boldf}{\heiti}
\newcommand\keywords[1]{\boldf{关键词:} \normf #1}

\begin{document}
	
	% 插入头部信息
	\maketitle
	% 换页
	\thispagestyle{empty}
	\clearpage
	
	% 插入目录、图、表并换页
	\tableofcontents
	\listoffigures
	\listoftables
	\setcounter{page}{1}
	% 罗马字母形式的页码
	\pagenumbering{roman}
	\clearpage
	% 从该页开始计数
	\setcounter{page}{1}
	% 阿拉伯数字形式的页码
	\pagenumbering{arabic}
	
	\section{\normf{双目相机测量模型}}
	\normf
	问题描述:
	在某时刻,有两相机同时观测到某个世界坐标系下的特征点$p$,其在左相机$C_1$的归一化像素坐标为(测量值):
	\begin{equation*}
		p^{C_1}=\begin{pmatrix}
			u^{C_1}&v^{C_1}
		\end{pmatrix}^T
	\end{equation*}
	其在右相机$C_2$的归一化像素坐标(测量值)为:
	\begin{equation*}
		p^{C_2}=\begin{pmatrix}
			u^{C_2}&v^{C_2}
		\end{pmatrix}^T
	\end{equation*}
	且已知两相机之间的位姿变换关系矩阵$R_{C_1}^{C_2}$和$r_{C_1}^{C_2}$、	相机1相对于e系的初始位姿变换关系矩阵$R_{C_1}^{e}$和$r_{C_1}^{e}$。设特征点$p$在两相机坐标系下的坐标为:
	\begin{equation*}
		\begin{aligned}
			r^{C_1}=\begin{pmatrix}
				X^{C_1}&Y^{C_1}&Z^{C_1}
			\end{pmatrix}^T\\
		r^{C_2}=\begin{pmatrix}
			X^{C_2}&Y^{C_2}&Z^{C_2}
		\end{pmatrix}^T
		\end{aligned}
	\end{equation*}
	特征点$p$的初始e系坐标为:
	\begin{equation*}
		\begin{aligned}
			r^e=\begin{pmatrix}
				X^{e}&Y^{e}&Z^{e}
			\end{pmatrix}
		\end{aligned}
	\end{equation*}
那么,有如下的关系式:
\begin{equation*}
	\begin{aligned}
			\begin{cases}
			r^{e}=R_{C_1}^{e}r^{C_1}+r_{C_1}^e\\		r^{C_1}=R_{C_2}^{C_1}r^{C_2}+r_{C_2}^{C_1}	\end{cases}\\
		\to\begin{cases}
			r^{C_1}=[R_{C_1}^e]^T(r^e-r_{C_1}^e)\\	
			r^{C_2}=R_{C_1}^{C_2}(r^{C_1}-r^{C_1}_{C_2})	\end{cases}
	\end{aligned}
\end{equation*}
	定义如下的误差方程:
	\begin{equation*}
		\begin{pmatrix}
			e^{C_1}\\e^{C_2}
		\end{pmatrix}=
		\begin{pmatrix}
			e^{C_1}_u\\e^{C_1}_v\\e^{C_2}_u\\e^{C_2}_v
		\end{pmatrix}=\begin{pmatrix}
		\frac{X^{C_1}}{Z^{C_1}}\\\frac{Y^{C_1}}{Z^{C_1}}
		\\\frac{X^{C_2}}{Z^{C_2}}\\\frac{Y^{C_2}}{Z^{C_2}}
	\end{pmatrix}-
		\begin{pmatrix}
			u^{C_1}\\v^{C_1}\\u^{C_2}\\v^{C_2}
		\end{pmatrix}+	\begin{pmatrix}
		n^{C_1}_u\\n^{C_1}_v\\n^{C_2}_u\\n^{C_2}_v
	\end{pmatrix}
	\end{equation*}
	我们的优化目标参数为$R_{C_1}^e$(用轴角$\theta_{C_1}^e$来表示)、$r_{C_1}^e$、$r^{e}$。所以,我们的雅克比矩阵为:
	\begin{equation*}
		J=\begin{pmatrix}
			\frac{\partial e^{C_1}}{\partial\theta_{C_1}^e}&
			\frac{\partial e^{C_1}}{\partial r_{C_1}^e}&
			\frac{\partial e^{C_1}}{\partial r^{e}}\\
			\frac{\partial e^{C_2}}{\partial\theta_{C_1}^e}&
			\frac{\partial e^{C_2}}{\partial r_{C_1}^e}&
			\frac{\partial e^{C_2}}{\partial r^{e}}
		\end{pmatrix}_{4\times 9}
	\end{equation*}
	其中第一行的偏导数为:
	\begin{equation*}
		\begin{aligned}
			\to\frac{\partial e^{C_1}}{\partial\theta_{C_1}^e}=&
			\frac{\partial e^{C_1}}{\partial r^{C_1}}\cdot			\frac{\partial r^{C_1}}{\partial \theta_{C_1}^e}\\=&
			\frac{1}{Z^{C_1}}\begin{pmatrix}
				1&0&-\frac{X^{C_1}}{Z^{C_1}}\\
				0&1&-\frac{Y^{C_1}}{Z^{C_1}}
			\end{pmatrix}\cdot
		(r^{C_1}\times)
		\end{aligned}
	\end{equation*}
\begin{equation*}
	\begin{aligned}
		\to\frac{\partial e^{C_1}}{\partial r_{C_1}^e}=&
		\frac{\partial e^{C_1}}{\partial r^{C_1}}\cdot			\frac{\partial r^{C_1}}{\partial r_{C_1}^e}\\=&
		\frac{1}{Z^{C_1}}\begin{pmatrix}
			1&0&-\frac{X^{C_1}}{Z^{C_1}}\\
			0&1&-\frac{Y^{C_1}}{Z^{C_1}}
		\end{pmatrix}\cdot
		(-[R_{C_1}^e]^T)
	\end{aligned}
\end{equation*}
\begin{equation*}
	\begin{aligned}
		\to\frac{\partial e^{C_1}}{\partial r^e}=&
		\frac{\partial e^{C_1}}{\partial r^{C_1}}\cdot			\frac{\partial r^{C_1}}{\partial r^e}\\=&
		\frac{1}{Z^{C_1}}\begin{pmatrix}
			1&0&-\frac{X^{C_1}}{Z^{C_1}}\\
			0&1&-\frac{Y^{C_1}}{Z^{C_1}}
		\end{pmatrix}\cdot
		[R_{C_1}^e]^T
	\end{aligned}
\end{equation*}
	第二行的偏导数为:
	\begin{equation*}
		\begin{aligned}
			\to\frac{\partial e^{C_2}}{\partial\theta_{C_1}^e}=&
			\frac{\partial e^{C_2}}{\partial r^{C_2}}\cdot	
			\frac{\partial r^{C_2}}{\partial  r^{C_1}}\cdot		\frac{\partial r^{C_1}}{\partial \theta_{C_1}^e}\\=&
			\frac{1}{Z^{C_2}}\begin{pmatrix}
				1&0&-\frac{X^{C_2}}{Z^{C_2}}\\
				0&1&-\frac{Y^{C_2}}{Z^{C_2}}
			\end{pmatrix}\cdot R_{C_1}^{C_2}\cdot
			(r^{C_1}\times)
		\end{aligned}
	\end{equation*}
	\begin{equation*}
		\begin{aligned}
			\to\frac{\partial e^{C_2}}{\partial r_{C_1}^e}=&
			\frac{\partial e^{C_2}}{\partial r^{C_2}}\cdot	
			\frac{\partial r^{C_2}}{\partial  r^{C_1}}\cdot		\frac{\partial r^{C_1}}{\partial r_{C_1}^e}\\=&
			\frac{1}{Z^{C_2}}\begin{pmatrix}
				1&0&-\frac{X^{C_2}}{Z^{C_2}}\\
				0&1&-\frac{Y^{C_2}}{Z^{C_2}}
			\end{pmatrix}\cdot R_{C_1}^{C_2}\cdot
			(-[R_{C_1}^e]^T)
		\end{aligned}
	\end{equation*}
	\begin{equation*}
	\begin{aligned}
		\to\frac{\partial e^{C_2}}{\partial r^e}=&
		\frac{\partial e^{C_2}}{\partial r^{C_2}}\cdot	
		\frac{\partial r^{C_2}}{\partial  r^{C_1}}\cdot		\frac{\partial r^{C_1}}{\partial r^e}\\=&
		\frac{1}{Z^{C_2}}\begin{pmatrix}
			1&0&-\frac{X^{C_2}}{Z^{C_2}}\\
			0&1&-\frac{Y^{C_2}}{Z^{C_2}}
		\end{pmatrix}\cdot R_{C_1}^{C_2}\cdot
		[R_{C_1}^e]^T
	\end{aligned}
\end{equation*}
\end{document}

