\documentclass[12pt, twocolumn]{article}

% 引入相关的包
\usepackage{amsmath, listings, fontspec, geometry, graphicx, ctex, color, subfigure}

% 设定页面的尺寸和比例
\geometry{left = 1.5cm, right = 1.5cm, top = 1.5cm, bottom = 1.5cm}

% 设定两栏之间的间距
\setlength\columnsep{1cm}

% 设定字体,为代码的插入作准备
\newfontfamily\ubuntu{Ubuntu Mono}

% 头部信息
\title{\normf{这是标题}}
\author{\normf{陈烁龙}}
\date{\normf{\today}}

% 代码块的风格设定
\lstset{
	language=C++,
	basicstyle=\scriptsize\ubuntu,
	keywordstyle=\textbf,
	stringstyle=\itshape,
	commentstyle=\itshape,
	numberstyle=\scriptsize\ubuntu,
	showstringspaces=false,
	numbers=left,
	numbersep=8pt,
	tabsize=2,
	frame=single,
	framerule=1pt,
	columns=fullflexible,
	breaklines,
	frame=shadowbox, 
	backgroundcolor=\color[rgb]{0.97,0.97,0.97}
}

% 字体族的定义
\newcommand{\normf}{\kaishu}
\newcommand{\boldf}{\heiti}
\newcommand\keywords[1]{\boldf{关键词:} \normf #1}
\newcommand{\lieal[1]}{\left\lfloor #1 \right\rfloor_\times}

\begin{document}
	
	% 插入头部信息
	\maketitle
	% 换页
	\thispagestyle{empty}
	\clearpage
	
	% 插入目录、图、表并换页
	\tableofcontents
	\listoffigures
	\listoftables
	\setcounter{page}{1}
	% 罗马字母形式的页码
	\pagenumbering{roman}
	\clearpage
	% 从该页开始计数
	\setcounter{page}{1}
	% 阿拉伯数字形式的页码
	\pagenumbering{arabic}
	
	\section{\normf{概述}}
	\normf
	对于激光里程计而言,帧间匹配对于位姿估计是比较重要的。一般的帧间匹配有点到点的、点到线的、点到面的(三维空间)。其基本的步骤都是需要先进行数据关联,而后构建损失函数进行优化。而本次论文里的NDT方法,不需要进行数据关联,其通过将点云格网化,并建立对应的正太分布模型,在进行帧间位姿估计。
	
	\section{\normf{论文算法}}
	本次主要记录三维空间的NDT(论文里是二维的)。
	
	首先,对于拿到的第一帧点云,我们将其划分成小的规则格网(正方体)。然后对其里面的点$p_i,\quad i\in[0,n-1]$进行相关统计,得到其正太发布模型:
	\begin{equation*}
		\begin{cases}
			\begin{aligned}
				\bar{\boldsymbol{p}}&=\frac{1}{n}\sum_{i=0}^{n-1}{^{L_1}\boldsymbol{p}_i}\\
				\boldsymbol{\Sigma}&=\frac{1}{n}\sum_{i=0}^{n-1}(^{L_1}\boldsymbol{p}_i-\bar{\boldsymbol{p}})(^{L_1}\boldsymbol{p}_i-\bar{\boldsymbol{p}})^T
			\end{aligned}
		\end{cases}
	\end{equation*}
	那么,在这一个格网内的任意点,都可以计算得到其概率密度:
	\begin{equation*}
		\boldsymbol{P}(\boldsymbol{p})=\exp\left(-\frac{\left(\boldsymbol{p}-\bar{\boldsymbol{p}}\right)^T\boldsymbol{\Sigma}^{-1}\left(\boldsymbol{p}-\bar{\boldsymbol{p}}\right)}{2}\right)
	\end{equation*}
	当然,为了减轻离散化效应,我们可以让划分的格网有重叠。比如,对于规定边长为$l$的正方体,在其右方错开$\frac{l}{2}$的地方放一个正方体,在其前方错开$\frac{l}{2}$的地方放一个正方体,在其前右方错开$\frac{l}{2}$的地方放一个正方体,然后此结构在往上方$\frac{l}{2}$的地方再放这样的一个结构。那么对于中心小格网的点,其落入了8个规则格网中,计算的时候,其概率就为8个规则格网计算得到的概率的和。
	
	通过上文的处理,我们进行对第一帧点云进行了建模。当第二帧到来的时候,我们通过一个两帧之间的位姿初值,将第二帧的点云里的每一个点${^{L_2}\boldsymbol{p}_j}$变换到第一帧坐标系中,得到${^{L_1}\boldsymbol{p}_j}$,然后计算每一个点的概率$\boldsymbol{P}({^{L_1}\boldsymbol{p}_j})$,然后求和,得到总的分数$s$。让这个分数最大即可以优化位姿$^{L_1}_{L_2}\boldsymbol{R}$、$^{L_1}\boldsymbol{t}_{L_2}$:
	\begin{equation*}
		\begin{cases}
			\begin{aligned}
					s_j&=\exp\left( -\frac{\left({^{L_1}\boldsymbol{p}_j}-\bar{\boldsymbol{p}}\right)^T\boldsymbol{\Sigma}^{-1}\left({^{L_1}\boldsymbol{p}_j}-\bar{\boldsymbol{p}}\right)}{2} \right)\\
					{^{L_1}\boldsymbol{p}_j}&={^{L_1}_{L_2}\boldsymbol{R}}{^{L_2}\boldsymbol{p}_j}+{^{L_1}\boldsymbol{t}_{L_2}}
			\end{aligned}
		\end{cases}
	\end{equation*}
	通过利用高斯-牛顿发可以优化求解位姿。首先我们的损失函数为:
	\begin{equation*}
		f(\boldsymbol{x})_j=\frac{1}{s_j}
	\end{equation*}
	
	求解雅克比矩阵(姿态处求导采用右扰动):
	\begin{equation*}
		\begin{cases}
			\begin{aligned}
				&\begin{aligned}
					\frac{\partial f(\boldsymbol{x})_j}{\partial {^{L_1}\boldsymbol{p}_j}}=&\frac{\partial f(\boldsymbol{x})}{\partial s_j}\times\frac{\partial s_j}{\partial {^{L_1}\boldsymbol{p}_j}}\\
					=&s_j^{-1}			\times\left({^{L_1}\boldsymbol{p}_j}-\bar{\boldsymbol{p}}\right)^T\boldsymbol{\Sigma}^{-1}
				\end{aligned}\\
				&\begin{aligned}
					\frac{\partial {^{L_1}\boldsymbol{p}_j}}{\partial \delta {^{L_1}_{L_2}\boldsymbol{\theta}} }&=-{^{L_1}_{L_2}\boldsymbol{R}}\lieal[{^{L_2}\boldsymbol{p}_j}]\\
					\frac{\partial {^{L_1}\boldsymbol{p}_j}}{\partial {^{L_1}\boldsymbol{t}_{L_2}} }&=I
				\end{aligned}
			\end{aligned}
		\end{cases}
	\end{equation*}
	而后求解:
	\begin{equation*}
			\begin{cases}
			\begin{aligned}
				\boldsymbol{J_j}&=\left(\frac{\partial f(\boldsymbol{x})_j}{\partial {^{L_1}\boldsymbol{p}_j}}\right)\begin{pmatrix}
					-{^{L_1}_{L_2}\boldsymbol{R}}\lieal[{^{L_2}\boldsymbol{p}_j}]&\boldsymbol{I}
				\end{pmatrix}\\
				\boldsymbol{H}&=\sum_{j=0}^{m-1}\boldsymbol{J_j}^T\boldsymbol{J_j}\\
				\boldsymbol{g}&=-{j=0}^{m-1}\boldsymbol{J_j}f(\boldsymbol{x})_j\\
				\boldsymbol{H}\Delta \boldsymbol{x}&=g \quad \boldsymbol{x}=\boldsymbol{x}+\Delta\boldsymbol{x}
			\end{aligned}
		\end{cases}
	\end{equation*}
	需要注意的是,由于姿态求解我们使用了右扰动,所以姿态更新时,要右乘更新量。
\end{document}

