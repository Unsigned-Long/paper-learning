\documentclass[12pt, twocolumn]{article}

% 引入相关的包
\usepackage{amsmath, listings, fontspec, geometry, graphicx, ctex, color, subfigure}

% 设定页面的尺寸和比例
\geometry{left = 1.5cm, right = 1.5cm, top = 1.5cm, bottom = 1.5cm}

% 设定两栏之间的间距
\setlength\columnsep{1cm}

% 设定字体,为代码的插入作准备
\newfontfamily\ubuntu{Ubuntu Mono}

% 头部信息
\title{\normf{这是标题}}
\author{\normf{陈烁龙}}
\date{\normf{\today}}

% 代码块的风格设定
\lstset{
	language=C++,
	basicstyle=\scriptsize\ubuntu,
	keywordstyle=\textbf,
	stringstyle=\itshape,
	commentstyle=\itshape,
	numberstyle=\scriptsize\ubuntu,
	showstringspaces=false,
	numbers=left,
	numbersep=8pt,
	tabsize=2,
	frame=single,
	framerule=1pt,
	columns=fullflexible,
	breaklines,
	frame=shadowbox, 
	backgroundcolor=\color[rgb]{0.97,0.97,0.97}
}

% 字体族的定义
\newcommand{\normf}{\kaishu}
\newcommand{\boldf}{\heiti}
\newcommand\keywords[1]{\boldf{关键词:} \normf #1}
\newcommand{\lieal[1]}{\left\lfloor #1 \right\rfloor_\times}

\begin{document}
	
	% 插入头部信息
	\maketitle
	% 换页
	\thispagestyle{empty}
	\clearpage
	
	% 插入目录、图、表并换页
	\tableofcontents
	\listoffigures
	\listoftables
	\setcounter{page}{1}
	% 罗马字母形式的页码
	\pagenumbering{roman}
	\clearpage
	% 从该页开始计数
	\setcounter{page}{1}
	% 阿拉伯数字形式的页码
	\pagenumbering{arabic}
	
	\section{\normf{概述}}
	\normf
	论文名字:《Robust Stereo Visual Inertial Odometry for Fast Autonomous Flight》,提出了一种基于滤波器的双目视觉加惯导的里程计算法。由于其主要应用于小型的飞行机器上,所以算法的实时性和精度是比较需要考量的地方。小型飞行器的挂载能力通常有限,不能使用比较大的传感器,所以视觉相机和IMU就成为了SLAM的首选传感器。
	
	另外,滤波的方式虽然略逊色于基于优化的方式,但是其计算量小,是比较合适的。至于选择使用双目相机,是因为其相比与单目相机,适应场景更强,当然也会带来额外的计算量。
	
	该论文使用了紧耦合的方式组织惯导系统和视觉系统。
	
	\section{\normf{数学推导}}
	\subsection{\normf{系统状态向量和状态误差向量}}
	首先,定义状态向量:
	\begin{equation*}
		\boldsymbol{x}_I=\begin{pmatrix}
			^{I}_{G}\boldsymbol{q}^T & \boldsymbol{b}_g^T&^{G}\boldsymbol{v}_I^T&\boldsymbol{b}_a^T&^{G}\boldsymbol{p}_I^T
			& ^{I}_{C}\boldsymbol{q}^T &^{I}\boldsymbol{p}_C^T
		\end{pmatrix}^T
	\end{equation*}
	其中:$^{I}_{G}\boldsymbol{q}^T$表示从$\boldsymbol{G}$系(全局坐标系,论文里是惯性坐标系)到$\boldsymbol{I}$系(IMU的坐标系,由于载体坐标系和IMU的坐标系重合,所以就是载体坐标系)的姿态变换四元素;$\boldsymbol{b}_g^T$表示陀螺的零偏;$^{G}\boldsymbol{v}_I^T$表示载体坐标系想对于惯性坐标系的速度向量;$\boldsymbol{b}_a^T$表示加速度计的零偏;$^{G}\boldsymbol{p}_I^T$表示载体坐标系想对于惯性坐标系的位置向量;$^{I}_{C}\boldsymbol{q}^T$表示相机坐标系想对于载体坐标系的姿态四元素;$^{I}\boldsymbol{p}_C^T$表示相机坐标系想对于载体坐标系的位置向量。
	
	而后,定义状态的误差向量:
	\begin{equation*}
		\tilde{\boldsymbol{x}}_I=\begin{pmatrix}
			^{I}_{G}\tilde{\boldsymbol{\theta}}^T & \tilde{\boldsymbol{b}}_g^T&^{G}\tilde{\boldsymbol{v}}_I^T&\tilde{\boldsymbol{b}}_a^T&^{G}\tilde{\boldsymbol{p}}_I^T
			& ^{I}_{C}\tilde{\boldsymbol{\theta}}^T &^{I}\tilde{\boldsymbol{p}}_C^T
		\end{pmatrix}^T
	\end{equation*}
	注意,这里使用轴角来替换四元素表示姿态,由一个四维向量变成了三维向量。这里的误差是这么定义的:
	\begin{enumerate}
		\item 姿态:
		\begin{equation*}
			\begin{aligned}
					\tilde{\boldsymbol{q}}&=\delta\boldsymbol{q}=\boldsymbol{q}\otimes \hat{\boldsymbol{q}}^{-1}\\
					\tilde{\boldsymbol{q}}&=\delta\boldsymbol{q}\approx\begin{pmatrix}
						\frac{1}{2}^{G}_{I}\tilde{\boldsymbol{\theta}}^T&1
					\end{pmatrix}^T
			\end{aligned}
		\end{equation*}
	这是因为对于轴角$\boldsymbol{\theta}$而言,其构成的四元素为:
	\begin{equation*}
		\boldsymbol{q}=\begin{pmatrix}
			\frac{\sin\Vert\frac{1}{2} \boldsymbol{\theta}\Vert}{\Vert\frac{1}{2} \boldsymbol{\theta}\Vert}\frac{1}{2}\boldsymbol{\theta}\\\cos\Vert\frac{1}{2} \boldsymbol{\theta}\Vert
		\end{pmatrix}
	\end{equation*}
当向量$\boldsymbol{\theta}$是小角度时,可以得到上文中的近似(类似于取极限)。
	\item 其他状态向量
	
	对于其他状态向量,其误差向量可以表示为(以$^{G}\boldsymbol{p}_I^T$为例):
	\begin{equation*}
		^{G}\tilde{\boldsymbol{p}}_I^T=^{G}\boldsymbol{p}_I^T-^{G}\hat{\boldsymbol{p}}_I^T
	\end{equation*}
	\end{enumerate}
	
	上文描述的就是IMU的状态向量,而相机的状态误差向量则为:
	\begin{equation*}
		\tilde{\boldsymbol{x}}_{C_i}=\begin{pmatrix}
			^{C_i}_{G}\tilde{\boldsymbol{\theta}}^T &^{G}\tilde{\boldsymbol{p}}_{C_i}^T
		\end{pmatrix}^T
	\end{equation*}
	注意,相机的位姿在同一个状态解算中有多个。综上,整个系统的状态误差向量为:
	\begin{equation*}
		\tilde{\boldsymbol{x}}=\begin{pmatrix}
			\tilde{\boldsymbol{x}}_I^T&\tilde{\boldsymbol{x}}_{C_1}^T&\tilde{\boldsymbol{x}}_{C_2}^T&\dots&\tilde{\boldsymbol{x}}_{C_N}^T
		\end{pmatrix}^T
	\end{equation*}
	\subsection{\normf{系统的过程模型}}
	该部分主要的目的是得到在EKF中预测部分的状态转移方程和误差模型,因此,在预测步骤中,我们主要基于的是IMU的状态。首先是IMU状态的微分方程,其描述了IMU状态随时间变化的规律:
	\begin{equation*}
		\begin{cases}
				\begin{aligned}
				^{I}_{G}\dot{\hat{\boldsymbol{q}}}&=\frac{1}{2}\Omega(\hat{\boldsymbol{\omega}})^{I}_{G}\hat{\boldsymbol{q}}\\
				\dot{\hat{\boldsymbol{b}}}_g&=\boldsymbol{0}_{3\times1}\\
				^{G}\dot{\hat{\boldsymbol{v}}}&=C(^{I}_{G}\hat{\boldsymbol{q}})^T\hat{\boldsymbol{a}}+^{G}\boldsymbol{g}\\
				\dot{\hat{\boldsymbol{b}}}_a&=\boldsymbol{0}_{3\times1}\\
				^{G}\dot{\hat{\boldsymbol{p}}}_I&=^{G}\hat{\boldsymbol{v}}\\
				^{I}_{C}\dot{\hat{\boldsymbol{q}}}&=\boldsymbol{0}_{3\times1}\\
				^{I}\dot{\hat{\boldsymbol{p}}}_C&=\boldsymbol{0}_{3\times1}
			\end{aligned}
		\end{cases}
	\end{equation*}
	其中:$C(\star)$表示将四元素表示的姿态变换成旋转矩阵。另外:
	\begin{equation*}
		\begin{cases}
			\begin{aligned}
				\hat{\boldsymbol{\omega}}&=\boldsymbol{\omega}_m-\hat{\boldsymbol{b}}_g\\
				\hat{\boldsymbol{a}}&=\boldsymbol{a}_m-\hat{\boldsymbol{b}}_a\\
				\Omega(\hat{\boldsymbol{\omega}})&=\begin{pmatrix}
					-\lieal[\hat{\boldsymbol{\omega}}]&\hat{\boldsymbol{\omega}}\\-\hat{\boldsymbol{\omega}}^T&0
				\end{pmatrix}
			\end{aligned}
		\end{cases}
	\end{equation*}
	上两个式子表示将IMU器件的偏差从原始测量值中移除。于是,我们可以得到如下的微分方程:
	\begin{equation*}
		\begin{cases}
		\begin{aligned}
		\dot{\tilde{\boldsymbol{x}}}_I&=\boldsymbol{F}\tilde{\boldsymbol{x}}_I+\boldsymbol{G}\boldsymbol{n}_I\\
		\boldsymbol{n}_I&=\begin{pmatrix}
			\boldsymbol{n}_g^T&\boldsymbol{n}_{\omega g}^T&\boldsymbol{n}_a^T&\boldsymbol{n}_{\omega a}^T
		\end{pmatrix}^T
		\end{aligned}
		\end{cases}
	\end{equation*}
	其中:$\boldsymbol{n}_g$表示陀螺的高斯噪声,$\boldsymbol{n}_a$表示加速度计的高斯噪声,$\boldsymbol{n}_{wg}$表示陀螺的随机游走速率,$\boldsymbol{n}_{wa}$表示加速度计的随机游走速率。具体系数矩阵$\boldsymbol{F}$的和$\boldsymbol{G}$形式请查看论文附录A。比如,$\boldsymbol{F}$的第三行第一列的矩阵块为:$-C(^{I}_{G}\hat{\boldsymbol{q}})^T\lieal[\hat{\boldsymbol{a}}]$,推导:
	\begin{equation*}
		\begin{aligned}
			^{G}\dot{{\boldsymbol{v}}}-^{G}\dot{\tilde{\boldsymbol{v}}}&=C(^{I}_{G}\hat{\boldsymbol{q}})^T(\boldsymbol{a}_m-({\boldsymbol{b}}_a-\tilde{\boldsymbol{b}}_a))+^{G}\boldsymbol{g}\\
			\to^{G}\dot{\tilde{\boldsymbol{v}}}&=-C(^{I}_{G}\hat{\boldsymbol{q}})^T(\boldsymbol{a}_m+\tilde{\boldsymbol{b}}_a-{\boldsymbol{b}}_a)-^{G}\boldsymbol{g}\\
			\to\frac{\partial ^{G}\dot{\tilde{\boldsymbol{v}}}}{\partial ^{I}_{G}\tilde{\boldsymbol{\theta}}}&=-\boldsymbol{I}\cdot C(^{I}_{G}\hat{\boldsymbol{q}})^T\lieal[\boldsymbol{a}_m+\tilde{\boldsymbol{b}}_a-{\boldsymbol{b}}_a]
			\\&=-C(^{I}_{G}\hat{\boldsymbol{q}})^T\lieal[\hat{\boldsymbol{a}}]\\
			\to\frac{\partial ^{G}\dot{\tilde{\boldsymbol{v}}}}{\partial \tilde{\boldsymbol{b}}_a}&=-C(^{I}_{G}\hat{\boldsymbol{q}})^T
		\end{aligned}
	\end{equation*}
	这样,也得到了$\boldsymbol{F}$的第三行第四列的矩阵块(求导的地方使用了李代数左乘扰动模型,即如何处理包含姿态的矩阵对姿态的求导)。
	
	得到连续的状态微分方程之后,我们需要对其进行离散化,才能在实际工程上应用。也就是我们要求解上面的微分方程。利用四阶Runge-Kutta方法,可以得到\footnote{\normf{推导过程可以参考吴云老师的《最优估计》教材第三章。}}:
	\begin{equation*}
		\begin{cases}
				\begin{aligned}
			\boldsymbol{\Phi}_k&=\boldsymbol{\Phi}(t_{k+1},t_k)=e^{\int_{t_k}^{t_{k+1}}F(\tau)d\tau}\\
			\boldsymbol{Q}_k&=\int_{t_k}^{t_{k+2}}\boldsymbol{\Phi}(t_{k+1},\tau)\boldsymbol{G}\boldsymbol{Q}\boldsymbol{G}^T\boldsymbol{\Phi}(t_{k+1},\tau)^Td\tau
		\end{aligned}	
		\end{cases}
	\end{equation*}
其中$\boldsymbol{\Phi}(t_{k+1},t_k)$即为从时刻$t_k$到时刻$t_{k+1}$的状态转移矩阵。而$\boldsymbol{Q}_k$即为系统的噪声协方差矩阵。根据卡尔曼滤波器的状态预测步骤,我们对上一时刻的状态进行预测,得到当前时刻的状态和其协方差矩阵:
	\begin{equation*}
	\begin{cases}
		\begin{aligned}
			\tilde{\boldsymbol{x}}_{I_{k+1|k}}&=\boldsymbol{\Phi}(t_{k+1},t_k)\tilde{\boldsymbol{x}}_{I_{k}}\\
			\boldsymbol{P}_{II_{k+1|k}}&=\boldsymbol{\Phi}(t_{k+1},t_k)\boldsymbol{P}_{II_{k|k}}\boldsymbol{\Phi}(t_{k+1},t_k)^T+\boldsymbol{Q}_k
		\end{aligned}	
	\end{cases}
\end{equation*}
	最后,在预测部分,我们的协方差矩阵为(注意,相机状态与IMU状态之间的协方差矩阵块在预测时,需要作用一个系数矩阵):
	\begin{equation*}
		\boldsymbol{P}_{k+1|k}=\begin{pmatrix}
			\boldsymbol{P}_{II_{k+1|k}}&\boldsymbol{\Phi}_k\boldsymbol{P}_{IC_{k+1|k}}\\\boldsymbol{P}_{IC_{k+1|k}}^T\boldsymbol{\Phi}_k^T&\boldsymbol{P}_{CC_{k+1|k}}
		\end{pmatrix}
	\end{equation*}
当新的相机位姿到来时,我们需要纳入进来。状态向量部分的增加如前面所述,而协方差矩阵则为:
	\begin{equation*}
		\boldsymbol{P}_{k|k}=\begin{pmatrix}
			\boldsymbol{I}_{21+6N}\\\boldsymbol{J}
		\end{pmatrix}\boldsymbol{P}_{k|k}\begin{pmatrix}
		\boldsymbol{I}_{21+6N}\\\boldsymbol{J}
	\end{pmatrix}^T
	\end{equation*}
其中,$J$矩阵可以通过误差传播得到(关心的是新相机位姿和IMU位姿之间的关系):
	\begin{equation*}
	\begin{cases}
		\begin{aligned}
			{^{C}_{G}\hat{\boldsymbol{q}}}&={^{C}_{I}\hat{\boldsymbol{q}}}\otimes{^{I}_{G}\hat{\boldsymbol{q}}}\\
			{^{G}\hat{\boldsymbol{p}}_C}&={^{G}\hat{\boldsymbol{p}}_I}+C({^{I}_{G}\hat{\boldsymbol{q}}}^T){^{I}\hat{\boldsymbol{p}}_C}\\
			\frac{\partial {^{C}_{G}\hat{\boldsymbol{q}}}}{\partial \delta{^{I}_{G}\boldsymbol{\theta}}}&=C^T({^{I}_{G}\hat{\boldsymbol{q}}})\\
			\frac{\partial {^{G}\hat{\boldsymbol{p}}_C}}{\partial \delta{^{I}_{G}\boldsymbol{\theta}}}&=C({^{I}_{G}\hat{\boldsymbol{q}}}^T)\lieal[{^{I}\hat{\boldsymbol{p}}_C}]=C^T({^{I}_{G}\hat{\boldsymbol{q}}})\lieal[{^{I}\hat{\boldsymbol{p}}_C}]\\
			\frac{\partial {^{G}\hat{\boldsymbol{p}}_C}}{\partial {^{G}\hat{\boldsymbol{p}}_I}}&=\boldsymbol{I}_{3\times3}
		\end{aligned}	
	\end{cases}
\end{equation*}
	\subsection{\normf{测量模型}}
	测量模型参考李圣雨师兄的论文对应的目录文件($lishengyu$)。注意里面用到了一个技巧,在构建测量模型是,通过在方程上左乘特征点的左零空间,将特征点边缘化,以加速求解。
	
	\subsection{\normf{滤波更新的技巧}}
	在EKF中,每次得到新IMU数据,都会进行状态预测,但是测量更新并不是一直在做。有两种条件触发测量更新:
	\begin{enumerate}
		\item 当跟踪的特征丢失的时候,进行测量更新。这时该特征点对应的相机帧不再增加(因为已经跟踪丢失了,不像ORB-SLAM一样进行重定位操作),即使的进行测量更新可以修正窗口内的系统状态。
		\item 当窗口的大小到达限制上限。设定一个窗口大小,当实际的窗口大小到达限制大小时,意味着要丢弃部分相机帧,所以即使进行测量更新。
	\end{enumerate}
	至于如何丢弃相机帧,论文的做法是每隔一个更新步骤就移除两个相机帧。具体而言,基于窗口内倒数第二帧和倒数第二帧之间的运动,要么最老的一个相机帧被移除,要么倒数第二帧被移除。重复这样做两次,就可以移除掉两个帧。由于倒数第一帧(最新的一帧)保持着提取的特征点,所以不考虑移除。
	
	这么做是有现实背景的,当载体高速运动是,倒数第二帧和倒数第二帧之间的运动会很大,那么移除掉最老的一个相机帧;如果倒数第二帧和倒数第二帧之间的运动很小(比如飞行器载体悬停),那么移除掉倒数第二帧。
	
	
	
	
	
	
	
	
	
	
	
	
	
	
	
	
	
	
	
	
	
	
	
	
	
	
\end{document}

