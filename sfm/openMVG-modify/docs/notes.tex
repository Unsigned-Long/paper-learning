\documentclass[12pt, twocolumn]{article}

% 引入相关的包
\usepackage{amsmath, listings, fontspec, geometry, graphicx, ctex, color, subfigure}

% 设定页面的尺寸和比例
\geometry{left = 1.5cm, right = 1.5cm, top = 1.5cm, bottom = 1.5cm}

% 设定两栏之间的间距
\setlength\columnsep{1cm}

% 设定字体,为代码的插入作准备
\newfontfamily\ubuntu{Ubuntu Mono}

% 头部信息
\title{\normf{这是标题}}
\author{\normf{陈烁龙}}
\date{\normf{\today}}

% 代码块的风格设定
\lstset{
	language=C++,
	basicstyle=\scriptsize\ubuntu,
	keywordstyle=\textbf,
	stringstyle=\itshape,
	commentstyle=\itshape,
	numberstyle=\scriptsize\ubuntu,
	showstringspaces=false,
	numbers=left,
	numbersep=8pt,
	tabsize=2,
	frame=single,
	framerule=1pt,
	columns=fullflexible,
	breaklines,
	frame=shadowbox, 
	backgroundcolor=\color[rgb]{0.97,0.97,0.97}
}

% 字体族的定义
\newcommand{\normf}{\kaishu}
\newcommand{\boldf}{\heiti}
\newcommand\keywords[1]{\boldf{关键词:} \normf #1}

\begin{document}
	
	% 插入头部信息
	\maketitle
	% 换页
	\thispagestyle{empty}
	\clearpage
	
	% 插入目录、图、表并换页
	\tableofcontents
	\listoffigures
	\listoftables
	\setcounter{page}{1}
	% 罗马字母形式的页码
	\pagenumbering{roman}
	\clearpage
	% 从该页开始计数
	\setcounter{page}{1}
	% 阿拉伯数字形式的页码
	\pagenumbering{arabic}

	
	\section{\normf{OpenMVG相机位姿}}
	\normf
	\subsection{\normf{相机位姿表示}}
	在OpenMVG库中,相机的位姿是通过结构体$Pose3$来表示的,其由两个成员构成:表示位置的$center$和表示姿态的$rotation$,但是代码里并没有说明其位姿转换的方向(矩阵的含义)。	
	
	事实上,$center$等同于$^{W}\boldsymbol{p}_C$,$rotation$等同于$^{C}_{W}\boldsymbol{R}$,有:
	\begin{equation*}
		{^{C}\boldsymbol{p}_i}={^{C}_{W}\boldsymbol{R}}({^{W}\boldsymbol{p}_i}-{^{W}\boldsymbol{p}_C})
	\end{equation*}	
	即:
	\begin{equation*}
		\begin{cases}
			\begin{aligned}
				{^{C}\boldsymbol{p}_i}&={^{C}_{W}\boldsymbol{R}}{^{W}\boldsymbol{p}_i}+\boldsymbol{t}\\
				\boldsymbol{t}&=-{^{C}_{W}\boldsymbol{R}}{^{W}\boldsymbol{p}_C}
			\end{aligned}
		\end{cases}
	\end{equation*}	
	换句话说,我们可以通过$\left[ {^{C}_{W}\boldsymbol{R}}|\boldsymbol{t} \right] $将世界坐标系下的点投影到相机坐标系下。其逆变换为:
	\begin{equation*}
		\begin{cases}
			\begin{aligned}
				{^{C}_{W}\boldsymbol{R}}^{-1}{^{C}\boldsymbol{p}_i}-{^{C}_{W}\boldsymbol{R}}^{-1}\boldsymbol{t}&={^{W}\boldsymbol{p}_i}\\
				{^{C}_{W}\boldsymbol{R}}^{-1}({^{C}\boldsymbol{p}_i}-\boldsymbol{t})&={^{W}\boldsymbol{p}_i}
			\end{aligned}
		\end{cases}
	\end{equation*}
	\subsection{\normf{复合位姿}}
	另外,复合两个位姿:
	\begin{equation*}
		{^{C_i}_{C_j}\boldsymbol{T}}=\begin{bmatrix}
			{^{C_i}_{C_j}\boldsymbol{R}}&-{^{C_i}_{C_j}\boldsymbol{R}}{^{C_j}\boldsymbol{p}_{C_i}}\\
			\boldsymbol{0}_{1\times 3}&1
		\end{bmatrix}
	\end{equation*}
	\begin{equation*}
	{^{C_j}_{W}\boldsymbol{T}}=\begin{bmatrix}
		{^{C_j}_{W}\boldsymbol{R}}&-{^{C_j}_{W}\boldsymbol{R}}{^{W}\boldsymbol{p}_{C_j}}\\
		\boldsymbol{0}_{1\times 3}&1
	\end{bmatrix}
	\end{equation*}
	\begin{equation*}
	{^{C_i}_{C_j}\boldsymbol{T}}{^{C_j}_{W}\boldsymbol{T}}=\begin{bmatrix}
		{^{C_i}_{C_j}\boldsymbol{R}}{^{C_j}_{W}\boldsymbol{R}}&-{^{C_i}_{C_j}\boldsymbol{R}}{^{C_j}_{W}\boldsymbol{R}}{^{W}\boldsymbol{p}_{C_j}}-{^{C_i}_{C_j}\boldsymbol{R}}{^{C_j}\boldsymbol{p}_{C_i}}\\
		\boldsymbol{0}_{1\times 3}&1
	\end{bmatrix}
	\end{equation*}
	\begin{equation*}
	\begin{cases}
		\begin{aligned}
			{^{C_i}_{W}\boldsymbol{R}}&={^{C_i}_{C_j}\boldsymbol{R}}{^{C_j}_{W}\boldsymbol{R}}\\
			{^{C_i}_{W}\boldsymbol{R}}^{-1}&={^{W}_{C_j}\boldsymbol{R}}{^{C_j}_{C_i}\boldsymbol{R}}\\
			^{W}\boldsymbol{p}_{C_i}&={^{C_i}_{W}\boldsymbol{R}}^{-1}({^{C_i}_{C_j}\boldsymbol{R}}{^{C_j}_{W}\boldsymbol{R}}{^{W}\boldsymbol{p}_{C_j}}+{^{C_i}_{C_j}\boldsymbol{R}}{^{C_j}\boldsymbol{p}_{C_i}})\\
			&={^{W}\boldsymbol{p}_{C_j}}+{^{W}_{C_j}\boldsymbol{R}}{^{C_j}\boldsymbol{p}_{C_i}}\\
			&={^{W}\boldsymbol{p}_{C_j}}+{^{C_j}_{W}\boldsymbol{R}^{-1}}{^{C_j}\boldsymbol{p}_{C_i}}
		\end{aligned}
	\end{cases}
	\end{equation*}
	这也就是下面代码块的含义了:
		\begin{lstlisting}[label=code2,caption={\normf 位姿复合}]
Pose3 operator * ( const Pose3& P ) const
{
		return {
				rotation_ * P.rotation_,
				P.center_ + P.rotation_.transpose() * center_
		};
}
	\end{lstlisting}
	
	不过,这有什么含义呢?我们已知第$j$个坐标系相对于世界坐标系的位姿和第$j$个坐标系相对于第$i$个坐标系,那么,我们就可以复合得到第$i$个坐标系相对于世界坐标系的位姿。
	
	\subsection{\normf{归化参考坐标系}}
	另外,由于使用$Ceres$库进行解算时,网是不受控的(BA算法会整体调整网结构)。所以一般没有相机的位姿和世界坐标系对齐。但是在V-SLAM中,我们又一般都以第一帧相机作为世界坐标系参考。所以再完成SfM结算后,我们需要将所有的数据(相机位姿,控制点,路标等)转到第一帧坐标系下。如果用$c$表示世界到相机的变换,那么:
	\begin{equation*}
		{^{C_i}_{C_0}\boldsymbol{T}}={^{C_i}_{W}\boldsymbol{T}}\times{^{W}_{C_0}\boldsymbol{T}}={^{C_i}_{W}\boldsymbol{T}}\times{^{C_0}_{W}\boldsymbol{T}^{-1}}
	\end{equation*}
	也就是说,我们需要在每一个相机的位姿基础上右乘一个参考相机位姿的逆。
	对于路标而言:
	\begin{equation*}
		{^{C_0}\boldsymbol{p}_i}={^{C_0}_{W}\boldsymbol{T}}{^{W}\boldsymbol{p}_i}
	\end{equation*}
	也就是说,我们需要对路标左乘一个参考相机的位姿。
	
	好在上面的一切都已经有一个函数实现了,就是:
			\begin{lstlisting}[label=code2,caption={\normf 位姿复合}]
void ApplySimilarity
(	const geometry::Similarity3 & sim,
	SfM_Data & sfm_data,
	bool transform_priors	);
	\end{lstlisting}
	你要将哪个相机作为参考系,只需要把这个相机的位子传到sim参数里就行了。
	
\end{document}

